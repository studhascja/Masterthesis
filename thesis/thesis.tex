\documentclass[twoside=false, %  einseitiger "Druck" (= optimiert für Abgabe als PDF)
    % BCOR=15mm, % Bindekorrektur; für Ausdruck mit Klebebindung Kommentar entfernen und ggf. anpassen
    chapterprefix=false,
    headinclude=true,
    footinclude=false,
    pagesize,%         write pagesize to DVI or PDF
    fontsize=11pt,%             use this font size
    paper=a4,%          use ISO A4
    bibliography=totoc,%         write bibliography-chapter to table of contents
%    index=totoc,%         write index-chapter to table of contents (Index is not used here)
    cleardoublepage=plain,% \cleardoublepage generates pages with pagestyle empty
    headings=big,%       A4/B5
    listof=flat,%        improved list of tables
    numbers=noenddot
  ]{scrbook}

\usepackage[listings=false]{scrhack}  % die erwünschten Änderungen sind im Paket listings bereits integriert
\usepackage{enumitem}
\usepackage{amsfonts}
\usepackage[slantedGreek,sc]{mathpazo}  % Schriftart Palatino

% \usepackage{lmodern}    % statt mathpazo, falls CM Fonts verwendet werden sollen
%\usepackage{mathptmx}    % statt mathpazo, falls Times  verwendet werden soll
\usepackage[scaled=.95]{helvet}
\usepackage{courier}
% der DIV-Faktor darf erst NACH dem Laden der Schriften gesetzt werden, damit der Satzspiegel korrekt berechnet wird
% kann manuell nach beliebgen angepasst werden (vergrößert bzw. verkleinert den für Text genutzten Bereich auf der Seite,
% es gibt aber ggf. eine Warung des typearea-Pakets (die man ignorieren kann)
\KOMAoptions{DIV=14}  % DIV Faktor für Satzspiegelberechnung, siehe Doku zu KOMA Script; 14 für Druck ohne Bindekorrektur, 15 mit Korrektur
\usepackage[T1]{fontenc}
\usepackage{textcomp}
\usepackage{mathtools}            % standard math notation (vectors/sets/...)
\usepackage{bm}        % standard math notation (fonts)
\usepackage{fixmath}        % standard math notation (fonts)
\usepackage{graphicx}
\usepackage[facing=yes]{floatrow}       % mehrere Gleitobjekte nebeneinander/caption neben Bild/Tabelle
\usepackage[labelfont=bf,sf,font=small,labelsep=space,format=plain]{caption}
\usepackage{subcaption}
\usepackage{scrlayer-scrpage}
\usepackage{epstopdf}
\usepackage[english]{babel}			% change to English for English settings (headings, table/figure names etc.) ALSO CHANGE selectlanguage command below!
%\usepackage[english]{babel}    % for English settings
\usepackage{listings}
\usepackage{ellipsis}  % Korrigiert den Weißraum um Auslassungspunkte
\usepackage{microtype}  % optischer Randausgleich etc.

\usepackage{xcolor}         % z.B. für schattierte Boxen
\usepackage{framed}			% shaded Umgebung
\definecolor{shadecolor}{gray}{.85}%
\usepackage{minted}
% Links im PDF
\usepackage[colorlinks=false,
            pdfborder={0 0 0},
            breaklinks=true,
            pdfpagelabels=true]
            {hyperref}

% Einstellungen für Bild-/Tabellenbeschriftung neben dem Bild
\floatsetup[figure]{capbesideposition={inside,top}}
\floatsetup[table]{capbesideposition={inside,top},style=plaintop}
\renewfloatcommand{fcapside}{figure}[\capbeside][\FBwidth]
\newfloatcommand{tcapside}{table}[\capbeside][\FBwidth]

% Einstellung für Bezeichnung des Quellcode-Verzeichnises
% comment out for English settings
%\renewcommand{\lstlistlistingname}{Code-Verzeichnis}


\deffootnote{1em}{1em}{%
 \makebox[1em][l]{\thefootnotemark}}

\newcommand{\real}{\mathord{\mathrm{I\!R}}}

\begin{document}
\lstset{language=java}
%\selectlanguage{ngerman}    
\selectlanguage{english}    % for English settings
\def\figdir{figures}
\def\tabledir{tables}

\frontmatter
\hypersetup{pageanchor=false}

\pagestyle{scrplain}
\pagestyle{empty}

\begin{titlepage}

\sffamily

\raggedleft

\vspace*{-2cm}

\includegraphics{\figdir/logo-th-rosenheim-2019_master_quer_2c.eps}

\vfill

\centering
\LARGE
% \vspace*{\fill}
%-----------
Faculty of computer science  \vspace{0.5cm}\\
\Large
Course of study Master of computer science

\vspace{2cm}

\LARGE

Wi-Fi for time critical applications: Practical measurements and evaluation of influencing variables on real time capability

\vspace{2cm}

\Large
Master Thesis

\vspace{1.5cm}


\Large
by

\vspace{0.5cm}

%\vspace*{\fill}

\LARGE
Jakob Hasche \vspace{1cm}

\vspace{1cm}

\flushleft
 \Large
\vspace*{\fill}

%-----------
\begin{tabbing}
Date of submission: \= tt.mm.jjjj \kill
Date of submission: \> tt.mm.jjjj \\
First examiner: \> Prof.\ Dr.\ Wolfgang Mühlbauer\\
Second examiner: \> Prof.\ Dr.\ Florian Künzner
\end{tabbing}
%-----------

\end{titlepage}

\cleardoubleemptypage

{
\large
\thispagestyle{empty}
\vspace*{\fill}

\noindent
\textsc{Eigenständigkeitserklärung / Declaration of Originality}

\medskip

\noindent
Hiermit bestätige ich, dass ich die vorliegende Arbeit selbständig verfasst und keine anderen als die angegebenen Hilfsmittel benutzt habe. Die Stellen der Arbeit, die dem Wortlaut oder dem Sinn nach anderen Werken (dazu zählen auch Internetquellen) entnommen sind, wurden unter Angabe der Quelle kenntlich gemacht.

\medskip

\textit{I declare that I have authored this thesis independently, that I have not used other than the declared sources / resources, and that I have explicitly marked all material which has been quoted either literally or by content from the used sources.}

\bigskip

\noindent
Rosenheim, the \today

\vspace*{2cm}

\noindent
Jakob Hasche
}

%%% Local Variables: 
%%% mode: latex
%%% TeX-master: "d"
%%% End: 

\cleardoubleemptypage
\include{abstract}
\cleardoubleemptypage

\pagestyle{scrplain}
\pagenumbering{roman}
\tableofcontents
\clearpage
\listoffigures
\clearpage
\listoftables
\clearpage          % entfernen, wenn kein Code-Verzeichnis erwünscht
\lstlistoflistings  % entfernen, wenn kein Code-Verzeichnis erwünscht
\cleardoublepage


\pagestyle{scrheadings}


\addtokomafont{caption}{\small}

\mainmatter
\hypersetup{pageanchor=true}
\chapter{Introduction}
\section{Problem description}
\section{Objective}
\section{Motivation}
\section{Layout of the work}
\chapter{Basics}
\section{Used Hardware}
\section{Yocto Linux}
\section{Wi-Fi standards}
\subsection{Quality of Service}
\subsection{Scheduling}
\subsection{IEEE 802.11n}
\subsection{IEEE 802.11ac}
\subsection{IEEE 802.11ax}
\subsection{IEEE 802.11be}
\section{Medium Access Mechanisms}
\section{Disruptive factors in Wi-Fis}
\section{Real Time Capability of Wi-Fi}
\chapter{Related Work}
\section{RT-Wi-Fi}
\section{Quality of Service in Wi-Fi}
\section{Disruptive factors in Wi-Fi's}
\section{Categorization of this work}
\chapter{Test Environment}
\section{Test concept}
\subsection{Requirements}
\subsection{Used Technologies}
\subsection{Server and Client Implementation}
\subsection{Monitoring}
\subsection{Test Automatization}
\subsection{Access point configuration}
\section{Operating System}
\subsection{Requirements}
\begin{itemize}
    \item As less background noise on the system as possible
    \item Capable for hard real time
    \item For my test environment the programming languages Python, Rust and the bpftrace library
    \item Capability to use the Wi-Fi Card
    \item Network tools, to manage my Wi-Fi interface and generate the workload
\end{itemize}

\subsection{Implementation}

\begin{itemize}
    \item local.conf \begin{itemize}
        \item set machine to raspberrypi 5
        \item set variable $RUST\_LIBC$
        \item set preferred linux version to 6.12
        \item activated license synaptics-killswitch
    \end{itemize}
    \item distro (version 5.2.1 on the walnascar branch) \begin{itemize}
        \item for mt7925e is a kernelversion of at least 5.7 necessary
        \item walnascar was the newest one to the time of june 2025
        \item CONFIG\_MT7925E=m
    \end{itemize}
    \item image \begin{itemize}
        \item hostapd
        \item dnsmasq
        \item python3, python3-pip
        \item net-tools, iproute2, wpa-supplicant, iw und networkmanager (Als Netzwerktools um interfaces, etc steuern zu können)
        \item werden mittels $IMAGE\_INSTALL$ hinzugefügt
        \end{itemize}
    \item Other Layers \begin{itemize}
        \item Poky
        \item meta-openembedded (oe, python, networking)
        \item meta-raspberrypi
        \item meta-rust in layer.conf the branch walnascar had to be added and the $RUST\_LIBC$ has to replaced with $\$\{@d.getVar('RUST\_LIBC').lower()\}$, and the variable had to be set in the local.conf
    \end{itemize}
    \item Own layer \begin{itemize}
        \item bb and bbappend files like normal
        \item higher priority
        \item set layer compatability to walnascar
    \end{itemize}
    \item RT-Patch \begin{itemize}
        \item from meta-raspberrypi layer only Linux 6.12.1 is given
        \item the normal patch is 6.12.28, this is not working
        \item from older patches 6.12.8 is working, but QA had to be added
        \item also the kernel config needed to be updated, especially the preemt rt and rcu boost, to read copy update and synchronization in the kernel for rt behaviour
    \end{itemize}
    \item device-tree overlay \begin{itemize}
        \item download pcie-32bit-dma-pi5.dtbo from https://github.com/raspberrypi/firmware/tree/master/boot/overlays
        \item write own recipe which installs the file at /boot/overlays/
        \item License file has to be created, because it's an own recipe, so the md5 checksum has to be added
        \item since Distro 5.1 for S ${WORKDIR}/sources$ has to be set, in this path the files are saved
        \item with do install first the folder is created and then the file is saved in the temporary root file system D with the rights 644
        \item With Files:\${PN} the file is set from the temporary file system to the final one 
        \item in the local.conf with $RPI\_USE\_OVERLAYS$ and $RPI\_EXTRA\_CONFIG$ the overlay is added to the config.txt
        \item in meta-raspberrypi/conf/machine/include/rpi-base.inc it has to be added in the $RPI\_KERNEL\_DEVICETREE\_OVERLAYS$ 
    \end{itemize}
    \item Custom Files \begin{itemize}
        \item hostapd \begin{itemize}
            \item bbappend in recipes-connectivity/hostapd
            \item all configs are located in /etc/hostapd
            \item same procedure like for the device tree overlay
            \item files are not located in workdir, but in workdir/sources-unpack
        \end{itemize}
        \item Rust Server + BPF \begin{itemize}
            \item Create bitbake blueprint with cargo bitbake git clone https://github.com/meta-rust/cargo-bitbake.git
            \item Add License
            \item set always to newest commit SRCREV = "\${AUTOREV}"
            \item Adjust workdir with S = "\${WORKDIR}/git/code/server"
            \item It depends on DEPENDS (Build-time dependency) += "clang-native kernel-devsrc pkgconfig-native zlib elfutils bpftool-native"
            \item (Run Time Dependency) RDEPENDS\_\${PN} += "libbpf"
            \item in a do\_compile\:prepand() $export KERNEL\_HEADERS="\${STAGING\_KERNEL\_DIR}",  VMLINUX_PATH="\${TOPDIR}/tmp/work/raspberrypi5-poky-linux/linux-raspberrypi/6.12.1+git/linux-raspberrypi5-standard-build/", build vm linux with    \$BPFT btf dump file "\${VMLINUX\_PATH}/vmlinux" format c > \${S}/src/bpf/vmlinux.h, preopare Clan and Bindgen, so that clang ist the compiler and bindgen hast the opportunity to find vmlinux.h export CC=clang export BINDGEN_EXTRA_CLANG_ARGS="--target=bpf -I\${S}/src/bpf", rust target bauen mit cargo build --release --target=\${TARGET_SYS} --target-dir=\${B}$
            \item add configs (add pahole to recipe)
            \item config can be seen in /yocto-rpi5/build/tmp/work/raspberrypi5-poky-linux/linux-raspberrypi/6.12.1+git/linux-raspberrypi5-standard-build/.conf 
            \item with bitbake -c menuconfig virtual/kernel the kernel config can be seen and also the depedencies of the single configs and which are forefilled and which not
            \item in source/.kernel-meta/cfg/merge\_config\_build.log can be seen if the configuration was noticed ant if it was really taken
            \item libbpf, clang, cargo, rust, libbpf-dev and elfutils had to be added to the image
            \item CONFIG\_COMPILE\_TEST=y CONFIG\_DEBUG\_INFO=y, CONFIG\_DEBUG\_INFO\_BTF=y, CONFIG\_FRAME\_POINTER=y, CONFIG\_DEBUG\_INFO\_DWARF4=y

        \end{itemize}
    \end{itemize}  
\end{itemize}

\begin{minted}[linenos, frame=lines, fontsize=\small]{python}
def hallo():
    print("Hallo Welt")
\end{minted}

\chapter{Wi-Fi Modifications}
\section{Quality of Service}
\section{Change of Wi-Fi Standards}
\section{Frequency}
\section{Bandwidth}

\chapter{Evaluation}
\section{Test procedure}
\section{Results of low disturbance test}
\section{Results of normal use test}
\section{Results of high disturbance test}
\section{Discussion}
\include{chapter_conclussion}
\appendix
\include{append}

\cleardoublepage

\bibliographystyle{natger}  % this is intended to be used together with German language settings; replace if you desire a different style
\bibliography{thesis}
\end{document}
