\chapter{Test Environment}
\section{Test concept}
\subsection{Requirements}
\subsection{Used Technologies}
\subsection{Server and Client Implementation}
\subsection{Monitoring}
\subsection{Test Automatization}
\subsection{Access point configuration}
\section{Operating System}
\subsection{Requirements}
\begin{itemize}
    \item As less background noise on the system as possible
    \item Capable for hard real time
    \item For my test environment the programming languages Python, Rust and the bpftrace library
    \item Capability to use the Wi-Fi Card
    \item Network tools, to manage my Wi-Fi interface and generate the workload
\end{itemize}

\subsection{Implementation}

\begin{itemize}
    \item local.conf \begin{itemize}
        \item set machine to raspberrypi 5
        \item set variable $RUST\_LIBC$
        \item set preferred linux version to 6.12
        \item activated license synaptics-killswitch
    \end{itemize}
    \item distro (version 5.2.1 on the walnascar branch) \begin{itemize}
        \item for mt7925e is a kernelversion of at least 5.7 necessary
        \item walnascar was the newest one to the time of june 2025
    \end{itemize}
    \item image \begin{itemize}
        \item hostapd
        \item dnsmasq
        \item python3, python3-pip
        \item net-tools, iproute2, wpa-supplicant, iw und networkmanager (Als Netzwerktools um interfaces, etc steuern zu können)
        \item werden mittels $IMAGE\_INSTALL$ hinzugefügt
        \end{itemize}
    \item Other Layers \begin{itemize}
        \item Poky
        \item meta-openembedded (oe, python, networking)
        \item meta-raspberrypi
        \item meta-rust in layer.conf the branch walnascar had to be added and the $RUST\_LIBC$ has to replaced with $\$\{@d.getVar('RUST\_LIBC').lower()\}$, and the variable had to be set in the local.conf
    \end{itemize}
    \item Own layer \begin{itemize}
        \item bb and bbappend files like normal
        \item higher priority
        \item set layer compatability to walnascar
    \end{itemize}
    \item RT-Patch \begin{itemize}
        \item from meta-raspberrypi layer only Linux 6.12.1 is given
        \item the normal patch is 6.12.28, this is not working
        \item from older patches 6.12.8 is working, but QA had to be added
        \item also the kernel config needed to be updated, especially the preemt rt and rcu boost, to read copy update and synchronization in the kernel for rt behaviour
    \end{itemize}
    \item device-tree overlay \begin{itemize}
        \item download pcie-32bit-dma-pi5.dtbo from https://github.com/raspberrypi/firmware/tree/master/boot/overlays
        \item write own recipe which installs the file at /boot/overlays/
        \item License file has to be created, because it's an own recipe, so the md5 checksum has to be added
        \item since Distro 5.1 for S ${WORKDIR}/sources$ has to be set, in this path the files are saved
        \item with do install first the folder is created and then the file is saved in the temporary root file system D with the rights 644
        \item With Files:\${PN} the file is set from the temporaray file system to the final one 
        \item in the local.conf with $RPI\_USE\_OVERLAYS$ and $RPI\_EXTRA\_CONFIG$ the overlay is added to the config.txt
        \item in meta-raspberrypi/conf/machine/include/rpi-base.inc it has to be added in the $RPI\_KERNEL\_DEVICETREE\_OVERLAYS$ 
    \end{itemize}
    \item Custom Files \begin{itemize}
        \item hostapd \begin{itemize}
            \item bbappend in recipes-connectivity/hostapd
            \item all configs are located in /etc/hostapd
            \item same procedure like for the device tree overlay
            \item files are not located in workdir, but in workdir/sources-unpack
        \end{itemize}
    \end{itemize}  
\end{itemize}